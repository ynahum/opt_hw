\documentclass[12pt]{article}
\usepackage[utf8]{inputenc}
\usepackage{amsmath}
\usepackage{amssymb}
\usepackage{graphicx}
\graphicspath{ {./plots/} }

\newcommand{\rectres}[1]{
\begin{center}
\begin{tabular}{ |c| }
\hline
 #1\\
\hline
\end{tabular}
\end{center}
}

\newcommand{\qed}{\hfill$\blacksquare$}

\title{Introduction to Numerical Optimization\\Assignment 4}
\author{Yair Nahum 034462796\\and\\blabla 11111111 }

\begin{document}

\maketitle

%\tableofcontents{}

\section{Constrained Optimization and Duality}

\subsection{Question 1}
We need to solve the following 
\begin{equation}
\label{eq:exp}
\begin{split}
    \min _x x^T M x + c^T x \\
    \text{s.t. } Ax = b
\end{split}
\end{equation}
Given:
\begin{itemize}
  \item $M \in \mathbb{R}^{n\times n},M \succ 0$
  \item $A \in \mathbb{R}^{m\times n}$
  \item $x,c \in \mathbb{R}^{n}$
  \item $b \in \mathbb{R}^{m}$
  \item Assuming $M$ and $AM^{-1}A^T$ are invertible.
\end{itemize}
The lagrangian is:
$$L(x,u)=x^T M x + c^T x + u^T(Ax-b)$$
We require the first KKT condition $\nabla_x L(x,u)=0$:
$$\nabla_x L(x,u)=0 = (M+M^T)x + c + A^Tu \Rightarrow$$
$$x^* = -(M+M^T)^{-1}(c + A^Tu)$$
The $(M+M^T)$ is invertible since it is P.D. ( $M$ is P.D. $\Rightarrow M^T$ is P.D $\Rightarrow (M+M^T)$ P.D.) and therefore it's invertible.\\
We denote $B=M+M^T$
The third KKT condition gives $h_i(x)=Ax-b=0$, if we put the $x^*$ in it we can issolate and get $u^*$:\\
$$-AB^{-1}(c + A^Tu) = b \Rightarrow$$
$$b+AB^{-1}c = -AB^{-1}A^Tu$$
The second KKT condition $0 = u^Th_i(x) = u^T(Ax-b)$:\\

\end{document}

