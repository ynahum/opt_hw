\documentclass{article}
\usepackage[utf8]{inputenc}
\usepackage{amsmath}
\usepackage{amssymb}
\usepackage{graphicx}
\graphicspath{ {./plots/} }

\newcommand{\rectres}[1]{
\begin{center}
\begin{tabular}{ |c| }
\hline
 #1\\
\hline
\end{tabular}
\end{center}
}

\newcommand{\qed}{\hfill$\blacksquare$}

\title{Introduction to Numerical Optimization\\HW2}
\author{Yair Nahum 034462796\\and\\blabla 11111111 }

\begin{document}

\maketitle

%\tableofcontents{}

\section{Convex Sets and Functions}

\subsection{Question 1}
Need to prove that if we have 2 convex functions $f_1$ and $f_2$ over convex set $C$, than $$g(x)=\max\limits_{i=1,2} f_i(x)$$ is also a convex function over convex set $C$
\subsubsection*{Proof}
Let's pick some points $s,t \in C$\\
We need to show that $$g(\alpha s + (1-\alpha)t) \leq \alpha g(s) + (1-\alpha)g(t) \hspace{20} \forall \alpha \in [0,1]$$
$f_1$ is convex $\Rightarrow$
$$f_1(\alpha s + (1-\alpha)t) \leq \alpha f_1(s) + (1-\alpha)f_1(t)$$
$f_2$ is convex $\Rightarrow$
$$f_2(\alpha s + (1-\alpha)t) \leq \alpha f_2(s) + (1-\alpha)f_2(t)$$
Since each function is convex in that intermediate point $(\alpha s + (1-\alpha)t)\in C$, we can only increment each max argument and deduce:
$$g(\alpha s + (1-\alpha)t) = \max\limits_{i=1,2} f_i(\alpha s + (1-\alpha)t) \leq \max\limits_{i=1,2} \{\alpha f_i(s) + (1-\alpha)f_i(t)\} \leq $$
$$\max\limits_{i=1,2} \{ \alpha f_i(s) \} + \max\limits_{i=1,2}\{(1-\alpha)f_i(t)\} = \alpha \max\limits_{i=1,2} f_i(s) + (1-\alpha)\max\limits_{i=1,2} f_i(t) = \alpha g(s) + (1-\alpha)g(t)$$
\qed\\
Another way to prove it is by contradiction:
Assume $g(x)$ is not convex. Meaning, $$\exists s,t \in C \text{ and } \exists \alpha \in [0,1] \text{  s.t.}$$ 
$$g(\alpha s + (1-\alpha)t) > \alpha g(s) + (1-\alpha)g(t)$$
On the other hand, $\forall i \in \{1,2\}$ 
$$\alpha g(s) + (1-\alpha)g(t) = \alpha \max\limits_{j=1,2} f_j(s) + (1-\alpha)\max\limits_{j=1,2} f_j(t) \geq \alpha  f_i(s) + (1-\alpha)f_i(t) \geq f_i(\alpha s + (1-\alpha)t)$$
When the last inequality is due to convexity of $f_i$
So using our assumption and the above, we get that $\forall i \in \{1,2\}$:
$$g(\alpha s + (1-\alpha)t) > f_i(\alpha s + (1-\alpha)t)$$
In contradiction to the definition of $g(x)$ that must be equal to one of the $f_i$ functions as:
$$g(x)=\max\limits_{i=1,2} f_i(x)$$
Therefore $g(x)$ must be convex.

\qed\\
\subsection{Question 2}
Let $f(x)$ be a convex function defined over convex domain $C$\\
We need to show that the level set $L = \{x \in C, f(x) \leq \alpha \}$ is a convex set.
Meaning, we need to prove that for any $y,z \in L$:
$$\beta y + (1-\beta)z \in L \:\:\:, \forall \beta \in [0,1]$$
That is:\\
(1)$\:\:\:\beta y + (1-\beta)z \in C$\\
$\text{ and }$\\
(2)$\:\:\:f(\beta y + (1-\beta)z) \leq \alpha$\\
\\
(1) From definition of $L$ and $C$:
$$\:\:\: y,z \in L \Rightarrow y,z \in C \Rightarrow \beta y + (1-\beta)z \in C$$
(2) Due to $f$ convexity over $C$ (first inequality) and $y,z \in L$ (second inequality):
$$f(\beta y + (1-\beta)z) \leq \beta f(y) + (1-\beta)f(z) \leq  \beta \alpha  + (1-\beta)\alpha = \alpha$$
\qed\\

\newpage
\subsection{Question 3}

\end{document}

